%%% Local Variables:
%%% mode: latex
%%% TeX-master: "program-analysis"
%%% End:

\chapter{Semantics}

\section{Simple Imperative Language}
\label{sec:simple-language}


\begin{figure}[h]
  \begin{math}
    \begin{array}{lcll}
      n & \in & \mathbb{V} & \text{Scalar values}\\
      x & \in & \mathbb{X} & \text{Program variables}\\
      \odot & = & + | - | * | \dots & \text{Binary operators}\\
      \oslash & = & < | \leq | == | \dots & \text{Comparison operators}\\
      E & = & & \text{Scalar expressions}\\
        & | & n & \text{Scalar constant}\\
        & | & x & \text{Variable}\\
        & | & E \odot E & \text{Binary operation}\\
      B & = & & \text{Boolean expressions}\\
        & | & x \oslash n & \text{Comparison of a variable with a constant}\\
      C & = & & \text{Commands (or Statements)}\\
        & | &\mathbf{skip} & \text{No-Op}\\
        & | & C; C & \text{Sequence}\\
        & | & x := E & \text{Assignment}\\
        & | & \mathbf{input}(x) & \text{Reading a value from input}\\
        & | & \mathbf{if}(B)\{C\} \mathbf{else} \{C\} & \text{Conditional statement}\\
        & | & \mathbf{while}(B)\{C\} & \text{Loop statement}\\
      P & = & C & \text{Program}\\
    \end{array}
  \end{math}

  \caption{Syntax of a simple imperative language}
  \label{fig:syntax}
\end{figure}