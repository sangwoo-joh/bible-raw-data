\chapter{Pointer Analysis}
\label{chap:pointer-analysis}

%%%%%%%%%%%%%%%%%%%%%%%%%%%%%%%%%%%%%%%%%%%%%%%%%%%%%%%%%%%%%%%%%%%%%%%%%%%%%%%%%%%%%%%%%%%%%
\section{Introduction}

\textit{Pointer analysis} or \textit{Points-to Analysis} is a static
program analysis that determines information on the values of pointer
variables or expressions.  Such information offers a static model of a
program's \textbf{heap}.  Since the heap is the primary structure for
\textit{global program data}, pointer analysis forms the substrate of
most \textbf{inter-procedural} static analyses.


The axes of \textit{precision} and \textit{performance} characterize
every approach to program reasoning. All interesting questions about
universal (i.e., all-inputs) program behaviour are
\textbf{undecidable}. We can use these axes to guide a more general
overview of the landscape of techniques for reasoning about program
memory. Further along the precision axis lie several approaches such
as \textit{shape analysis} and \textit{separation logic}. Separation
logic is a full-fledged logic, typically deep in the forests of
undecidability, where reasoning requires close human guidance. Shape
analysis is computable, automated program analysis, yet with
performance complexity in the territory of intractability: complexity
bounds for the best-known shape analyses are super-exponential.


%%%%%%%%%%%%%%%%%%%%%%%%%%%%%%%%%%%%%%%%%%%%%%%%%%%%%%%%%%%%%%%%%%%%%%%%%%%%%%%%%%%%%%%%%%%%%
\section{Core Pointer Analysis}

The standard formulation of a pointer analysis is as a computation of
the set of objects (the \textit{points-to set}) that a program
variable may point to during runtime.


The conventional approach for abstraction of heap objects is to
represent objects as \textit{allocation sites}, i.e., to consider a
\textbf{single abstract object} to stand in for each run-time object
allocated by the same instruction. The precision for the number of
memory allocation invocation is typically \textit{lost} in the
approximation performed by the analysis that is done statically.


\subsection{Andersen-Style Points-To Analysis, Declaratively}

The best-known and most straightforward pointer analysis algorithm is
Andersen-style, which can be easily expressed as \textit{subset
  constraints}: different program statements induce inferences of the
form ``points-to set A is a subset of points-to set B'', or,
computationally, ``add all elements from points-to set A to points-to
set B''.


Computation in Datalog consists of monotonic logical inferences that
repeatedly apply to produce more facts until a fixpoint is reached. A
Datalog rule ``$C(z,x) \gets A(x,y), B(y,z)$'' means that if $A(x,y)$
and $B(y,z)$ are both true, for same values $x,y,z$, then $C(z,x$) can
be inferred. Syntatically, the left arrow symbol ($\gets$) separates
the \textit{inferred} facts (the \textit{head} of the rule) from the
\textit{previously established} facts (the \textit{body} of the rule).


The target language is a simple intermediate language with the
followings:

\begin{itemize}
\item \texttt{Alloc}: allocating an object on heap. $ var = new ... $
\item \texttt{Move}: copying between local variables. $ to = from $
\item \texttt{Store}, \texttt{Load}: instructions for writing to the
  heap (or to object fields). $ to = base.fld $ and
  $ base.fld = from $
\item \texttt{Vcall}: virtually calling the method of the appropriate
  signature defined in the dynamic class of the receiver object.
  $ base.sig(...)$
\end{itemize}


Figure \ref{fig:pta-domain}, \ref{fig:input-relations}, and
\ref{fig:computed-relations} shows the domain\footnote{the different
  value sets that consistute the space of our computation} of the
analysis, its input relations, and the computed (intermediate and
output) relations respectively.

\paragraph{Input Relations}

The input relations correspond to the intermediate language. They are
logically grouped into relations that represent \textit{instructions}
and \textit{relations} that represent \textit{name-and-type}
information.

Similarily, there are relations that encode \textbf{type system},
\textbf{symbol table}, and \textbf{program environment}.
\begin{itemize}
\item \texttt{FormalArg}: shows which variable is a formal argument of
  a given method at a certain index, the \textit{n}-th argument.
\item \texttt{LookUp}: matches a method signature to the actual method
  definition inside a type.
\item \texttt{HeapType}: matches an object to its type, and is a
  function of its first argument.
\item \texttt{ActualReturn}: a function of its first argument (a
  method invocation site) and returns the local variable at the call
  site that receives the method call's return value.
\item \texttt{VarType}: maps a variable to its type.
\item \texttt{SubType}: links a type to its supertypes.
\item \texttt{InMethod}: function from instructions to their
  containing method.
\end{itemize}


\paragraph{Computed Relations}

The main output relations are \texttt{VarPointsTo} and
\texttt{CallGraph}, encoding our points-to and call-graph information.


\begin{figure}[ht]
  \centering

  \begin{math}
    \begin{array}{lcl}
      V & : & \text{a set of program variables} \\
      H & : & \text{a set of heap abstractions, i.e., the allocation sites} \\
      M & : & \text{a set of method identifiers, i.e., the function names} \\
      S & : & \text{a set of method signatures including name, type signature} \\
      F & : & \text{a set of fields} \\
      I & : & \text{a set of instructions} \\
      T & : & \text{a set of class types} \\
      \mathbb{N} & : & \text{the set of natural numbers}
    \end{array}
  \end{math}

  \caption{Domain}
  \label{fig:pta-domain}
\end{figure}


\begin{figure}

  \begin{math}
    \begin{array}{l}
      \mathtt{Alloc}(\mathtt{var}: V, \mathtt{heap}: H, \mathtt{inMeth}: M) \\
      \mathtt{Move}(\mathtt{to}: V, \mathtt{from}: V) \\
      \mathtt{Load}(\mathtt{to}: V, \mathtt{base}: V, \mathtt{fld}: F) \\
      \mathtt{Store}(\mathtt{base}: V, \mathtt{fld}: F, \mathtt{frome}: V) \\
      \mathtt{Vcall}(\mathtt{base}: V, \mathtt{sig}: S, \mathtt{invo}: I, \mathtt{inMeth}: M) \\
      \\
      \\
      \mathtt{FormalArg}(\mathtt{meth}: M, \mathtt{n}: \mathbb{N}, \mathtt{arg}: V) \\
      \mathtt{ActualArg}(\mathtt{invo}: I, \mathtt{n}: \mathbb{N}, \mathtt{arg}: V) \\
      \mathtt{FormalReturn}(\mathtt{meth}: M, \mathtt{ret}: V) \\
      \mathtt{ActualReturn}(\mathtt{invo}: I, \mathtt{var}: V) \\
      \mathtt{ThisVar}(\mathtt{meth}: M, \mathtt{this}: V) \\
      \mathtt{HeapType}(\mathtt{heap}: H, \mathtt{type}: T) \\
      \mathtt{LookUp}(\mathtt{type}: T, \mathtt{sig}: S, \mathtt{meth}: M) \\
      \mathtt{VarType}(\mathtt{var}: V, \mathtt{type}: T) \\
      \mathtt{InMethod}(\mathtt{instr}: I, \mathtt{meth}: M) \\
      \mathtt{Subtype}(\mathtt{type}: T, \mathtt{superT}: T) \\
    \end{array}
  \end{math}

  \caption{Input Relations}
  \label{fig:input-relations}
\end{figure}


\begin{figure}
  \begin{math}
    \begin{array}{l}
      \mathtt{VarPointsTo}(\mathtt{var}: V, \mathtt{heap}: H) \\
      \mathtt{CallGraph}(\mathtt{invo}: I, \mathtt{meth}: M) \\
      \mathtt{FldPointsTo}(\mathtt{baseH}: H, \mathtt{fld}: F, \mathtt{heap}: H) \\
      \mathtt{InterProcAssign}(\mathtt{to}: V, \mathtt{from}: V) \\
      \mathtt{Reachable}(\mathtt{meth}: M) \\
    \end{array}
  \end{math}

  \caption{Computed Relations}
  \label{fig:computed-relations}
\end{figure}


\paragraph{Analysis Logic}

Figure \ref{fig:andersen-logic} shows a complete Andersen-style
points-to analysis for intermediate language.

For example, the first rule states that, if we have computed that a
method is reachable and it contains the allocation of an abstract
object, \textit{heap}, directly assigned to local variable
\textit{var}, then \textit{var} is inferred to points to
\textit{heap}.

The analysis integreates two features that merit discussion, since
they represent standard variability axes in the literature:

\begin{itemize}
\item \textbf{Field sensitivity}: the ability of the analysis to
  distinguish different fields of the same abstract object, instead of
  lumping all fields together. This reflected in our ruels for
  handling \texttt{Store} and \texttt{Load} instructions: A
  \texttt{Store} input fact, together with prior \texttt{VarPointsTo}
  inferences, leads to computing a \texttt{FldPointsTo} relation fact
  for the given heap object and field. The \texttt{FldPointsTo} facts
  are later used in the handling of \texttt{Load} instructions, as the
  respective rule shows.

  The alternative would be a \textit{field-insensitive} analysis,
  which ignores which object fiels is used in those instructions.

  Yet another option is a \textit{field-based} analysis, which
  distinguishes fields but only identifies \texttt{FldPointsTo} facts
  by the heap object's type and not its full identity.

  Without extensive experimentation with real programs, it is hard to
  predict whether such tradeoffs will pay off. Performance often
  exhibits sharp discontinuities: the lack of precision may end up
  also hurting performance, since the manipulated points-to sets can
  be larget.

\item \textbf{On-the-fly call-graph construction}: the property that a
  points-to analysis also infers simultaneously which methods are
  called at each call-site. The two questions of ``what objects can a
  variable point to?'' and ``which function can get called at this
  call-site?'' are closely inter-related in any higher-order
  language. Generally, virtual calls in object-oriented languages and
  first-class functions in a functional language necessitate that
  call-graph computation have a model of points-to
  information. On-the-fly call-graph construction typically leads to a
  significantly more precise call-graph, which, in turn, enhances the
  precision of the points-to analysis. The two analyses are in a
  virtuous circle, with each helping the other achieve precision
  without wasting effort in spurious inferences.

  On-the-fly call-graph construction is reflected in the rule for
  \texttt{Vcall}. This is the most involved rule of Andersen-style
  algorithm. It states that, if the program has an instruction making
  a virtual method call over local variable \textit{base}, and if the
  computation so far has established that \textit{base} can point to
  heap object \textit{heap} in a reachable method, then the called
  method is looked up by-signature inside the type of \textit{heap}
  and several further facts are inferred: that the looked up method is
  \textbf{reachable}, that it has an edge in the call-graph from the
  current invocation site, and that its \textit{this} variable can
  point to \textit{heap}.

  Furthermore, the computed call-graph information is used to
  \textbf{flow} points-to information between actual and formal
  parameters of methods. For instance, if we have computed a
  call-graph edge between invocation site \textit{invo} and method
  \textit{meth}, then we infer an inter-procedural assignment to the
  \textit{i}-th formal argument of \textit{meth} from the
  \textit{i}-th actual argument at \textit{invo}, for every
  \textit{i}. Such assignments are treaed much like local assignments.

  Although on-the-fly call-graph construction is typically
  advantageous, there are several alternatives in the literature. A
  conservative \textit{a priori} call-graph computation has the
  advantage of speed, as well as of easier handling of complex
  language features (e.g., reflective calls). The techcniques of
  \textit{Rapid Type Analysis (RTA)} and \textit{Class Hierarchy
    Analysis (CHA)} are the most commonly used in this space. Both
  techniuqes employ only type information to determine an
  over-approximation of all methods potentially called at a site.
\end{itemize}


\begin{figure}[ht]
  \begin{math}
    \begin{array}[\textwidth]{lcl}
      \mathtt{VarPointsTo(var, heap)} & \gets & \mathtt{Reachable(meth), Alloc(var, heap, meth)} \\
      \\
      \mathtt{VarPointsTo(to, heap)} & \gets & \mathtt{Move(to, from), VarPointsTo(from, heap)} \\
      \\
      \mathtt{FldPointsTo(baseH, fld, heap)} & \gets & \mathtt{Store(base, fld, from), VarPointsTo(from, heap)} \\
      & & \mathtt{VarPointsTo(base, baseH)} \\
      \\
      \mathtt{VarPOintsTo(to, heap)} & \gets & \mathtt{Load(to, base, fld), VarPointsTo(base, baseH)} \\
      & & \mathtt{FldPointsTo(baseH, fld, heap)} \\
      \\
      \mathtt{Reachable(toMeth)},  & & \mathtt{Vcall(base, sig, invo, inMeth)}, \mathtt{Reachable(inMeth)}, \\
      \mathtt{VarPointsTo(this, heap)}, & \gets & \mathtt{VarPointsTo(base, heap)}, \mathtt{HeapType(heap, heapT)}, \\
      \mathtt{CallGraph(invo, toMeth)} & & \mathtt{LookUp(heapT, sig, toMeth), ThisVar(toMeth, this)} \\
      \\
      \mathtt{InterProcAssign(to, from)} & \gets & \mathtt{CallGraph(invo, meth), FormalArg(meth, n, to)} \\
      & & \mathtt{ActualArg(invo, n, from)} \\
      \\
      \mathtt{InterProcAssign(to, from)} & \gets & \mathtt{CallGraph(invo, meth), FormalReturn(meth, from), } \\
      & & \mathtt{ActualReturn(invo, to)} \\
      \\
      \mathtt{VarPointsTo(to, heap)} & \gets & \mathtt{InterProcAssign(to, from), VarPointsTo(from, heap)} \\
    \end{array}
  \end{math}

  \caption{Andersen Algorithm}
  \label{fig:andersen-logic}
\end{figure}


The analysis \textbf{loses precision} relative to an ideal result, but
this is hardly a surprise. Any algorithm for an undecidable algorithm
will sacrifice precision for computability. Notably, the algorithm is
still \textit{sound}, if a fact is true the algorithm will infer
it. Generally, pointer analysis is a \textit{may-analysis}: its
inferences intend to \textbf{over-approximate} actual program
behaviour. The opposite is \textit{must-analysis} that intends to
\textbf{under-approximate}.




\subsection{Other Approaches}

\paragraph{Steensgaard's Analysis}

Steensgaard-style pointer analysis is best termed
\textit{unification-based} and uses \textit{equality constraints} as
opposed to the subset constraints of the Andersen approach.

Steensgaard-style analyses have been quite popular, especially in
procedural languages like C, due to their simplicity and unparalleled
speed. However, they have become \textit{progressively less used} in
recent PL and moder settings, where the speed of an Andersen-style
analysis is usually quite sufficient.



\paragraph{Other Heap Abstractions}

Not just variables but entire expressions, or syntactic abstractions
over them, can be used as the root elements of an analysis. This
obviates the need to represent \textit{allocation sites} or
\textit{objects} explicitly.

The \textbf{access path} abstraction approach generalizes alias
relations by expressing all knowledge about the heap in terms of
relationships between pointer expressions. An access path is \textit{a
  variable qualifed by a sequence of field names}. Access paths can be
statically summarized syntactically in order to capture unbounded
runtime field chains. An analysis can use access paths internally or
in expressing its computed output.



%%%%%%%%%%%%%%%%%%%%%%%%%%%%%%%%%%%%%%%%%%%%%%%%%%%%%%%%%%%%%%%%%%%%%%%%%%%%%%%%%%%%%%%%%%%%%
%%% Local Variables:
%%% mode: latex
%%% TeX-master: "program-analysis"
%%% End:
