%%% Local Variables:
%%% mode: latex
%%% TeX-master: "program-analysis"
%%% End:

%%%%%%%%%%%%%%%%%%%%%%%%%%%%%%%%%%%%%%%%%%%%%%%%%%%%%%%%%%%%%%%%%%%%%%%%%%%%%%%%%%%%%%%%%%%%%%%%%%%
\chapter{Data Flow Analysis}
\label{chap:data-flow-analysis}

%%%%%%%%%%%%%%%%%%%%%%%%%%%%%%%%%%%%%%%%%%%%%%%%%%%%%%%%%%%%%%%%%%%%%%%%%%%%%%%%%%%%%%%%%%%%%%%%%%%
\section{Prerequisites}

\subsubsection{While-programs}

\begin{math}
  \begin{array}{lll}
    a & = & x \blank|\blank n \blank|\blank a_1 \mathtt{op}_a a_2 \\
    b & = & \mathtt{true} \blank|\blank \mathtt{false} \blank|\blank \mathtt{not} b \blank|\blank b_1 \mathtt{op}_b b_2 \blank|\blank a_1 \mathtt{op}_r a_2 \\
    S & = & [x := a]^l \blank|\blank [skip]^l \blank|\blank S_1; S_2 \blank|\blank \mathtt{if} \blank [b]^l \mathtt{then} \blank S_1 \blank \mathtt{else} \blank S_2 \blank|\blank \mathtt{while}\blank [b]^l \blank \mathtt{do} \blank S
  \end{array}
\end{math}


\begin{itemize}
\item Label $l$ denotes the program point (location of code).
\item $init(S)$ is the label of the first elementary bloc of $S$.
\item $final(S)$ is the set of labels of the last elementary blocks of
  $S$.
\item $labels(S)$ is the entire set of labels in the statement $S$.
\item Flows $flow(S)$ is the forward representation of how control
  flows in $S$.
\item Reverse flows $flow^R(S)$ is the backward representation of how
  control flows in $S$.
\item A statement consists of a set of \textit{elementary blocks}
  where $blocks : Stmt \to \mathcal{P}(Blocks)$
\item A statement $S$ is \textit{label consistent} iff any two
  elementary statements $[S_1]^l$ and $[S_2]^l$ with the same label in
  $S$ are equal: $S_1 = S_2$
\item A statement where all labels are unique is automatically label
  consistent.

\end{itemize}


%%%%%%%%%%%%%%%%%%%%%%%%%%%%%%%%%%%%%%%%%%%%%%%%%%%%%%%%%%%%%%%%%%%%%%%%%%%%%%%%%%%%%%%%%%%%%%%%%%%
\section{Intra-procedural Analysis}

\subsection{Available Expressions Analysis}

For each program point, which expressions must have already been
computed, and not later modified, on all paths to the program point?


\subsection{Reaching Definitions Analysis}

For each program point, which assignments may have been made and not
overwritten, when program execution reaches this point along some
path?

\subsection{Very Busy Expressions Analysis}

An expression is \textit{very busy} at the exit from a label if, no
matter what path is taken from the label, the expression is always
used before any of the variables occurring in it are re-defined.

The aim of the analysis is to determine for each program point, which
expressions must be very busy at the exit from the point.

\subsection{Live Variables Analysis}

A variable is \textit{live} at the exit from a label if there is a
path from the label to a \texttt{use} of the variable that does not
re-define the variable.

The aim of the analysis is to determine for each program point, which
variables may be live at the exit from the point.

\subsection{Derived Data Flow Information}

\begin{itemize}
\item Use-Def Chains: each \texttt{use} of a variable is linked to all
  \textbf{assignments} that reach it.
\item Def-Use Chains: each \texttt{assignment} to a variable is linked
  to all \texttt{use}s of it.
\end{itemize}


%%%%%%%%%%%%%%%%%%%%%%%%%%%%%%%%%%%%%%%%%%%%%%%%%%%%%%%%%%%%%%%%%%%%%%%%%%%%%%%%%%%%%%%%%%%%%%%%%%%
\section{Monotone Frameworks}

Each of the four classical analyses take the form

\begin{math}
  \begin{array}{lcl}
    \mathit{Analysis}_i(l) & = &
                                 \begin{cases}
                                   i & \mathtt{if} \blank l \in E \\
                                   \join \{ \mathit{Analysis}(l') | (l', l) \in F \} & otherwise
                                 \end{cases} \\
    \\
    \mathit{Analysis}(l) & = & f_l(\mathit{Analysis}_i(l)) \\
  \end{array}
\end{math}

where

\begin{itemize}
\item $\join$ is $\cap$ or $\cup$ (and $\meet$ is $\cup$ or $\cap$)
\item $F$ is either $flow(S_*)$ or $flow^R(S_*)$
\item $E$ is $\{ init(S_*) \}$ or $final(S_*)$
\item $i$ specified the initial or final analysis information
\item $f_l$ is the transfer function associated with $B^l \in blocks(S_*)$
\end{itemize}

\subsection{The Principles}

Forward vs. Backward

\begin{itemize}
\item \textbf{Forward analyses} have $F$ to be $flow(S_*)$ and then
  $\mathit{Analysis}_i$ concerns entry conditions and
  $\mathit{Analysis}$ concerns exit conditions; the equation system
  pre-supposes that $S_*$ has isolated entries.
\item \textbf{Backward analyses} have $F$ to be $flow^R(S_*)$ and then
  $\mathit{Analysis}_i$ concerns exit conditions and
  $\mathit{Analysis}$ concerns entry conditions; the equation system
  pre-supposes that $S_*$ has isolated exits.
\end{itemize}


Union vs. Intersection

\begin{itemize}
\item When $\join$ is $cap$, we require the \textbf{greatest sets}
  that solve the equations and we are able to detect properties
  satisfied by \textit{all execution paths} reaching (or leaving) the
  entry (or exit) of a label; the analysis is called a
  \textbf{must}-analysis.
\item When $\join$ is $\cup$, we require the \textbf{smallest sets}
  that solve the equations and we are able to detect properties
  satisfied by \textit{at least one execution path} to (or from) the
  entry (or exit) of a label; the analysis is called a
  \textbf{may}-analysis.
\end{itemize}


\subsection{Transfer Functions}

The set of transfer functions $\mathcal{F}$ is a set of
\textbf{monotone functions} over $L$, meaning that

\begin{math}
  \begin{array}{lcl}
    l \sle l' & \implies & f_l(l) \sle f_l(l')
  \end{array}
\end{math}

and furthermore they fulfill the following conditions:

\begin{itemize}
\item $\mathcal{F}$ contains \textit{all} the transfer functions $f_l : L \to L $ in question (for $l \in Lab_* $ )
\item $\mathcal{F}$ contains the \textit{identity function}
\item $\mathcal{F}$ is \textit{closed} under \textit{composition} of
  functions
\end{itemize}



%%%%%%%%%%%%%%%%%%%%%%%%%%%%%%%%%%%%%%%%%%%%%%%%%%%%%%%%%%%%%%%%%%%%%%%%%%%%%%%%%%%%%%%%%%%%%%%%%%%
\section{Equation Solving}

\subsection{The MFP Solution}

MFP stands for ``Maximum'' (actually least) Fixed Point.  The key idea
is to iterate until stabilisation.





\subsection{The MOP Solution}

MOP stands for ``Meet'' (actually join) Over all Paths.  The key idea
is to propagate analysis information along paths.


%%%%%%%%%%%%%%%%%%%%%%%%%%%%%%%%%%%%%%%%%%%%%%%%%%%%%%%%%%%%%%%%%%%%%%%%%%%%%%%%%%%%%%%%%%%%%%%%%%%

\section{Inter-procedural Analysis}


\subsection{The MVP Solution}

MVP stands for ``Meet'' over Valid Paths.

We need to match procedure entries and exits:

A \textit{complete path} from $l_1$ to $l_2$ in $P_*$ has proper
nesting of procedure entries and exits; and a procedure returns to the
point where it was called.



%%%%%%%%%%%%%%%%%%%%%%%%%%%%%%%%%%%%%%%%%%%%%%%%%%%%%%%%%%%%%%%%%%%%%%%%%%%%%%%%%%%%%%%%%%%%%%%%%%%
\section{Sensitivity}
\label{sec:sensitivity}

\subsection{Flow Sensitive}
\label{sec:flow-sensitive}

A \textbf{flow-sensitive} analysis takes into account the
\textit{order of statements} in a program.

\subsection{Path Sensitive}
\label{sec:path-sensitive}

A \textbf{path-sensitive} analysis computes different pieces of
analysis information dependent of the \textit{predicates} at
conditional branch instructions.

\subsection{Context Sensitive}
\label{sec:context-sensitive}

A \textbf{context-sensitive} analysis is an \textit{inter-procedural}
analysis that considers the \textit{calling context} when analyzing
the target of a function call. In particular, using context
information, one can \textit{jump back} to the original call site,
whereas without that information, the analysis information has to be
propagated back to all possible call sites, potentially losing
precision.
