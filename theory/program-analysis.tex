\documentclass[12pt, twopage]{book}
\usepackage{lmodern}
\usepackage[T1]{fontenc}     % font encoding
\usepackage[utf8]{inputenc}  % input encoding
\usepackage{hyperref}        % hyperlinks
\usepackage{graphicx}
\usepackage[english]{babel}
\usepackage{mathtools}
\usepackage{stmaryrd}

%%%%%%%%%%%%%%%%%%%%%%%%%%%%%%%%%%%%%%%%%%%%%%%%
% Chapter quote at the start of chapter        %
% Source: http://tex.stackexchange.com/a/53380 %
%%%%%%%%%%%%%%%%%%%%%%%%%%%%%%%%%%%%%%%%%%%%%%%%
\makeatletter
\renewcommand{\@chapapp}{}% Not necessary...
\newenvironment{chapquote}[2][2em]
  {\setlength{\@tempdima}{#1}%
   \def\chapquote@author{#2}%
   \parshape 1 \@tempdima \dimexpr\textwidth-2\@tempdima\relax%
   \itshape}
  {\par\normalfont\hfill--\ \chapquote@author\hspace*{\@tempdima}\par\bigskip}
\makeatother


\title{
  \Huge \textbf{Bible}
  \\
  \huge {Proram Analysis}
}

\author{\textsc{Sangwoo Joh}\footnote{\url{https://sangwoo-joh.github.io}}}


\begin{document}

\frontmatter
\maketitle

\tableofcontents  % auto generate index from chapter, section, subsection, ...
% \listoffigures  % auto generate list from figure's caption
% \listoftables   % auto generate list from tagble's caption

\mainmatter


%%%%%%%%%%%%%%%%%%%%%%%%%%%%%%%%%%%%%%%%%%%%%%%%%%%%%%%%%%%%%%%%%%%%%%%%%
\chapter*{Preface}
This is personal bible, notes for everything I've learnt.


%%%%%%%%%%%%%%%%%%%%%%%%%%%%%%%%%%%%%%%%%%%%%%%%%%%%%%%%%%%%%%%%%%%%%%%%%%%%%%%%%%%%%%%%%%%%%%%%%%%
\chapter{Program Analysis}
\begin{chapquote}{H.G. Rice [1953], \textit{paraphrased by Anders Moller}}
  ``Everything interesting about the behaviour of programs is
  undecidable.''
\end{chapquote}

The goal of \textit{static program analysis} is to verify specific
properties (or behaviours) of the target program \textbf{without its
  execution}.

For program \textsl{P} and property (specification) \textsl{S},

\begin{itemize}
\item $ \llbracket P \rrbracket $: Formal semantics of program \textsl{P}.

\item \textsl{S}: Semantic properties that we're interested in. This
  could be defined in various level, such as ``Division-by-zero will
  \textbf{never} occur'' or ``The variable \textit{i} is always 3''.

\item Soundness: $ analysis(P) = true \implies S $, in other words, if
  an analysis for program \textsl{P} says that it satisfies property
  \textsl{S}, then the program will truly satisfy that property.

\item Completeness: $ S \implies analysis(P) = true $, in other words,
  if a program satisfies a property \textsl{S}, then the analysis for
  that program says that it will satisfy that property.

\item Scalability: Time complexity.
\end{itemize}


\section{Semantics}
\label{sec:semantics}

\begin{itemize}
\item Operational Semantics: transitional
\item Denotational Semantics: compositional
\end{itemize}


%%%%%%%%%%%%%%%%%%%%%%%%%%%%%%%%%%%%%%%%%%%%%%%%%%%%%%%%%%%%%%%%%%%%%%%%%%%%%%%%%%%%%%%%%%%%%%%%%%%
\chapter{Abstract Interpretation}
\label{chap:ai}



\end{document}
